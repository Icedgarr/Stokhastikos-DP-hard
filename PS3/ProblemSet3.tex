\documentclass[11pt, english]{article}
\usepackage[english]{babel}
\usepackage[utf8]{inputenc}



\usepackage{geometry}
\geometry{
	a4paper,
	left=20mm,
	top=30mm,
	right=20mm
}


\usepackage{listings}
\usepackage{amsmath}
\usepackage{amsfonts}
\usepackage{amssymb}
\usepackage{amsthm} 
\usepackage{mathrsfs}
\usepackage{mathabx}
\usepackage{graphicx}
\usepackage{eurosym}
\usepackage{subfigure}
\usepackage{dsfont}
\usepackage{bbm}

%\lstdefinestyle{myCustomMatlabStyle}{
%	language=Matlab,
%	numbers=left,
%	stepnumber=1,
%	numbersep=10pt,
%	tabsize=4,
%	showspaces=false,
%	showstringspaces=false
%}

\usepackage{graphicx} 
\usepackage{fancyvrb} 
\usepackage{listings} 
\usepackage{listings}
\usepackage{amsmath}
\usepackage{amsfonts}
\usepackage{amssymb}
\usepackage{amsthm} 
\usepackage{mathrsfs}
\usepackage{mathabx}
\usepackage{graphicx}
\usepackage{eurosym}
\usepackage{subfigure}
\usepackage{dsfont}
\usepackage{bbm}
\usepackage{inputenc}

\lstdefinestyle{myCustomMatlabStyle2}{% setup listings 
	language=R,% set programming language 
	basicstyle=\small,% basic font style 
	keywordstyle=\bfseries,% keyword style 
	commentstyle=\ttfamily\itshape,% comment style 
	numbers=left,% display line numbers on the left side 
	numberstyle=\scriptsize,% use small line numbers 
	numbersep=10pt,% space between line numbers and code 
	tabsize=3,% sizes of tabs 
	showstringspaces=false,% do not replace spaces in strings by a certain character 
	captionpos=b,% positioning of the caption below 
	breaklines=true,% automatic line breaking 
	escapeinside={(*}{*)},% escaping to LaTeX 
	fancyvrb=true,% verbatim code is typset by listings 
	extendedchars=false,% prohibit extended chars (chars of codes 128--255) 
	literate={"}{{\texttt{"}}}1{<-}{{$\leftarrow$}}1{<<-}{{$\twoheadleftarrow$}}1 
	{~}{{$\sim$}}1{<=}{{$\le$}}1{>=}{{$\ge$}}1{!=}{{$\neq$}}1{^}{{$^\wedge$}}1,% item to replace, text, length of chars 
	alsoletter={.<-},% becomes a letter 
	alsoother={$},% becomes other 
	otherkeywords={!=, ~, $, *, \&, \%/\%, \%*\%, \%\%, <-, <<-, /},% other keywords 
	deletekeywords={c}% remove keywords 
}

\newcommand{\grafico}[5]{
	\begin{figure}
		[h!tbp]
		\centering
		\includegraphics[scale=#2, angle=#3]{#1}
		%\captionsetup{width=13cm}
		\caption{#4\label{#5}}
	\end{figure}
}

\newcommand{\su}[2]{\sum\limits_{#1}^{#2}}


\setlength{\parindent}{0pt}

\title{Stochastic Models and Optimization: Problem Set 3}
\author{Roger Garriga Calleja, Jos� Fernando Moreno Guti�rrez, David Rosenfeld, Katrina Walker}
\date{\today}

\begin{document}
	
\section{Q2. Inventory Pooling}
\textbf{Primitives}\\ D = demands \\ Q = quantity ordered \\ P = price \\ h =
inventory costs =  c-s \\ b = backholding costs = p-c \\ \\ First we will show
that $Q_p^* = \sqrt{n}Q^* + \mu(n- \sqrt{n})$%
\begin{align*} P (\sum_{L= 1}^n D_i \leq Q_p^*) = &  \frac{b}{b+R} \\ = &
P(\sqrt{n}D_i + \mu(n - \sqrt{n})\leq Q_p^*) = \frac{b}{b+R} \\ = & P(D_i \leq
\frac{1}{\sqrt{n}}(Q_p^*- \mu(n-\sqrt{n})))= \frac{b}{b + h} \end{align*} which
implies that this is equal to $Q^*$. Thus, we can use the hint to find the
following: 
\begin{align*} %Q^* = \frac{1}{\sqrt{n}}(Q_p^*- \mu(n-\sqrt{n}))
\end{align*}
\begin{align*} G(Q) = CQ - PD + h(Q-D)^++b(D-Q)^+ \\ G(Q) = CQ -
PE[D] + hE[Q-D]^+bE[D-Q]^+ \\ G'(Q^*)= C + hP(D \leq Q^*) - b(1-P[D\leq Q^*]) =
0 \\ P(D \leq Q^*) = \frac{-c+b}{h+b} \\ 
\end{align*} By using the hint, we can prove the desired result: 
\begin{align*} nG(Q^*) &=
n[CQ^*-PE[D]+hE[(Q^*-D)^+]+bE[(D-Q^*)^+]]\\ G(Q^*_p) & =
[CQ^*_p-PE[\sum_{i=1}^nD_i]+hE[(Q^*_p-\sum_{i=1}^nD_i)^+]+bE[(\sum_{i=1}^nD_i-Q^*_p)^+]]\\ 
\end{align*} 
Since $\sum_{i = 1}^nD_i = \sqrt{n}D_i+\mu(n-\sqrt{n})$ and $Q^*_P = \sqrt{n}Q^*+\mu(n-\sqrt{n})$, we can show that:\\
\begin{align*} 
G(Q^*_P) = & [C\sqrt{n}Q^*+\mu(n-\sqrt{n})-PE[\sqrt{n}D_i +  \mu(n-\sqrt{n})^+]\\
+& \ hE[(\sqrt{n}Q^*+\mu(n-\sqrt{n})-\sqrt{n}D_i-\mu(n-\sqrt{n}))^+] \\
+& b E[\sqrt{n}D_i+\mu(n-\sqrt{n})-\sqrt{n}Q^*+\mu(n-\sqrt{n})^+]]\\ 
\end{align*} 
\begin{align*} 
G(Q^*_P) = & [C\sqrt{n}Q^*+\mu(n-\sqrt{n})-PE[\sqrt{n}D_i +  \mu(n-\sqrt{n})^+]\\
+& \ hE[(\sqrt{n}Q^*-\sqrt{n}D_i] \\
+& b E[\sqrt{n}D_i-\sqrt{n}Q^*]]\\ 
\end{align*} 
\textbf{Problem 4 (An Investment Problem)}
\end{document}