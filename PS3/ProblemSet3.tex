\documentclass[11pt, english]{article}
\usepackage[english]{babel}
\usepackage[utf8]{inputenc}



\usepackage{geometry}
\geometry{
	a4paper,
	left=20mm,
	top=30mm,
	right=20mm
}


\usepackage{listings}
\usepackage{amsmath}
\usepackage{amsfonts}
\usepackage{amssymb}
\usepackage{amsthm} 
\usepackage{mathrsfs}
\usepackage{mathabx}
\usepackage{graphicx}
\usepackage{eurosym}
\usepackage{subfigure}
\usepackage{dsfont}
\usepackage{bbm}

%\lstdefinestyle{myCustomMatlabStyle}{
%	language=Matlab,
%	numbers=left,
%	stepnumber=1,
%	numbersep=10pt,
%	tabsize=4,
%	showspaces=false,
%	showstringspaces=false
%}

\usepackage{graphicx} 
\usepackage{fancyvrb} 
\usepackage{listings} 
\usepackage{listings}
\usepackage{amsmath}
\usepackage{amsfonts}
\usepackage{amssymb}
\usepackage{amsthm} 
\usepackage{mathrsfs}
\usepackage{mathabx}
\usepackage{graphicx}
\usepackage{eurosym}
\usepackage{subfigure}
\usepackage{dsfont}
\usepackage{bbm}
\usepackage{inputenc}

\lstdefinestyle{myCustomMatlabStyle2}{% setup listings 
	language=R,% set programming language 
	basicstyle=\small,% basic font style 
	keywordstyle=\bfseries,% keyword style 
	commentstyle=\ttfamily\itshape,% comment style 
	numbers=left,% display line numbers on the left side 
	numberstyle=\scriptsize,% use small line numbers 
	numbersep=10pt,% space between line numbers and code 
	tabsize=3,% sizes of tabs 
	showstringspaces=false,% do not replace spaces in strings by a certain character 
	captionpos=b,% positioning of the caption below 
	breaklines=true,% automatic line breaking 
	escapeinside={(*}{*)},% escaping to LaTeX 
	fancyvrb=true,% verbatim code is typset by listings 
	extendedchars=false,% prohibit extended chars (chars of codes 128--255) 
	literate={"}{{\texttt{"}}}1{<-}{{$\leftarrow$}}1{<<-}{{$\twoheadleftarrow$}}1 
	{~}{{$\sim$}}1{<=}{{$\le$}}1{>=}{{$\ge$}}1{!=}{{$\neq$}}1{^}{{$^\wedge$}}1,% item to replace, text, length of chars 
	alsoletter={.<-},% becomes a letter 
	alsoother={$},% becomes other 
	otherkeywords={!=, ~, $, *, \&, \%/\%, \%*\%, \%\%, <-, <<-, /},% other keywords 
	deletekeywords={c}% remove keywords 
}

\newcommand{\grafico}[5]{
	\begin{figure}
		[h!tbp]
		\centering
		\includegraphics[scale=#2, angle=#3]{#1}
		%\captionsetup{width=13cm}
		\caption{#4\label{#5}}
	\end{figure}
}

\newcommand{\su}[2]{\sum\limits_{#1}^{#2}}

\setlength{\parindent}{0pt}
\begin{document}
\title{Stochastic Models and Optimization: Problem Set 3}
\author{Roger Garriga Calleja, Jos� Fernando Moreno Guti�rrez, David Rosenfeld, Katrina Walker}
\date{\today}
\section{Introduction}
\section{Q2. Inventory control with forecasts}
\textbf{(a): Single period}\\
\textbf{Primitives}
$D$: Demand, which is a random variable \\ $D =     \left\{ \begin{array}{rcl}
         P_L & \mbox{ with probability q}\\ 
         P_S  & \mbox{with probability 1 - q}
                \end{array}\right.$\\
$p$: Sale price per unit\\ $c$: Ordering cost per unit\\$s$: Salvage value per unit\\ $b = p - c$: Backorder cost per unit\\ $h = c - s$: Inventory holding cost per unit\\ $Q$: Inventory decision.\\
\\
The profit for this problem can be written as:\\
$\pi(Q) = E[p.min\{D, Q\} - c.Q + s(Q - D)^+]$\\
$\pi(Q) = pE[D] - (p - c)E[(D - Q)^+] + (c - s)E[(Q - D)^+]$\\
$\pi(Q) = pE[D] - bE[(D - Q)^+] - hE[(Q - D)^+]$\\
\\
We then expand the formula to account for the expectation of demand D which takes value $P_L$ with probability $q$:\\
$$\pi(Q) = pE[D] - b[q.E_{P_L}[(D - Q)^+] + (1 -q).E_{P_S}[(D - Q)^+]] - h[qE_{P_L}[(Q - D)^+] + (1 - q)E_{P_S}[(Q - D)^+]]$$
\\
We take the derivative of the profit with respect to quantity in order to minimise, and set it equal to zero (with $P_L$ and $P_S$ the probability distributions in the large and small demand cases respectively):\\
$$\pi'(Q^*) = -b[qP_L(D > Q^*) + (1 -q)P_S(D > Q^*)] + h[qP_L(Q^* > D) + (1 - q)P_S(Q^* > D) = 0$$
\\
$$-bq(1 -P_L(D \leq Q^*) - b(1 -q)(1 -P_S(D \leq Q^*)) + hqP_L(D \leq Q^*) + h(1 - q)P_S(D \leq Q^*) = 0$$
We can then obtain that the optimal decision $Q^*$ satisfies:
$$qP_L(D \leq Q^*) + (1 - q)P_S(D \leq Q^*) = \frac{b}{b + h}$$


\textbf{(b) Multi-period problem}\\
\textbf{Primitives}\\
$x_k$: Inventory level at period k \\$u_k$: Decision variable over the number of units to order.\\ $w_k$: Demand at period k.\\ $y_{k+1} = \xi$ where $\xi$ takes the value "large demand" with probability q:\\ $\xi_k =     \left\{ \begin{array}{rcl}
         L & \mbox{ with probability q}\\ 
         S  & \mbox{with probability 1 - q}
                \end{array}\right.$\\
$c$: ordering cost per unit\\ $p$: Sale price per unit\\ $s$: Salvage value\\ $b = p - c$: Backorder cost per unit\\ $h = c - s$: Inventory holding cost per unit\\
\\
\textbf{Dynamics:}\\
$x_{k+1} = x_k + u_k - w_k$\\ $y_{k+1} = \xi_k$\\ $g_N(x_N) = -s(x_N)$\\ $g_k(x_k, u_k, w_k) = cu_k - pw_k + b max\{w_k - x_k - u_k, 0\} + h max\{x_k + u_k - w_k\}$\\
\\
\textbf{DP algorithm}:
$J_N(x_N) = -s(x_N)$\\
$J_k(x_k) = \min_{u_k \geq 0}E_{w_k, \xi_k}[g_k(x_k, u_k, w_k) + J_{k+1}(x_k, u_k, w_k)]$\\
\\
We can introduce the variable $z_k = x_k + u_k$, and rewrite our DP equation as follows:\\
$$J_k(x_k) = \min_{z_k \geq x_k}[G_k(z_k)] - cx_k$$
where $G_k(z_k) = cz_k + bE[max\{0, w_k - z_k\}] + hE[max\{0, z_k - w_k\}] + E[J_{k+1}(z_k - w_k)]$\\
\\
Once again we can expand this expression to take into account the two different values of $\xi$ for large and small demand:\\
$G_k(z_k) = cz_k + bqE[max\{0, w_k - z_k\} | \xi = L] + b(1 -q)E[max\{0, w_k - z_k\} | \xi = S] + hqE[max\{0, z_k - w_k\} | \xi = L] + h(1 -q)E[max\{0, z_k - w_k\} | \xi = S] + E[J_{k+1}(z_k - w_k)]$\\
\\
$G_k(z_k)$ is a convex function for all k, and there will be a solution $S_k$ which minimises $G(z)$ such that the optimal ordering policy becomes:\\
$u_k^* = \mu_k^* = S_k - x_k$ if $S_k > x_k$, and 0 otherwise.
\section{Q3. Inventory Pooling}
\textbf{Primitives}\\ D = demands \\ Q = quantity ordered \\ P = price \\ h =
inventory costs =  c-s \\ b = backholding costs = p-c \\ \\ First we will show that $Q_p^* = \sqrt{n}Q^* + \mu(n- \sqrt{n})$%
\begin{align*} 
G(Q) = hE[(Q-D)]^+bE[(D-Q)]^+ \\ G'(Q^*)= hP(D \leq Q^*) - b(1-P[(D\leq Q^*)]) =
0 \\ P(D \leq Q^*) = \frac{b}{h+b} \\ 
\end{align*}
Now, for the pooling we would obtain the same:
\begin{align*} 
P (\sum_{L= 1}^n D_i \leq Q_p^*) = &  \frac{b}{b+R} \\
\end{align*}
Since  this $\sum_{i = 1}^nD_i = \sqrt{n}D_i+\mu(n-\sqrt{n})$, 
\begin{align*}	
P(\sqrt{n}D_1 + \mu(n - \sqrt{n})\leq Q_p^*) = \frac{b}{b+R}  \Longleftrightarrow P(D_1 \leq
\frac{1}{\sqrt{n}}(Q_p^*- \mu(n-\sqrt{n})))= \frac{b}{b + h} \end{align*} 
which implies that $\frac{1}{\sqrt{n}}(Q_p^*- \mu(n-\sqrt{n})))$ = $Q^*$. Thus, $Q^*_P = \sqrt{n}Q^*+\mu(n-\sqrt{n})$.  Next, we will apply the hint to prove the desired result:
\begin{align*} nG(Q^*) &=
n[hE[(Q^*-D)^+]+bE[(D-Q^*)^+]]\\ G(Q^*_p) & =
[hE[(Q^*_p-\sum_{i=1}^nD_i)^+]+bE[(\sum_{i=1}^nD_i-Q^*_p)^+]]\\ 
\end{align*} 
Since $\sum_{i = 1}^nD_i = \sqrt{n}D_i+\mu(n-\sqrt{n})$ and $Q^*_P = \sqrt{n}Q^*+\mu(n-\sqrt{n})$, we can show that:\\
\begin{align*} 
G(Q^*_P) & =  [hE[(\sqrt{n}Q^*+\mu(n-\sqrt{n})-\sqrt{n}D_i-\mu(n-\sqrt{n}))^+] \\
+& b E[\sqrt{n}D_i+\mu(n-\sqrt{n})-\sqrt{n}Q^*+\mu(n-\sqrt{n})^+]]\\
& = \sqrt{n}hE[(Q^*-D_i)^+] + \sqrt{n} \ b E[(D_i-Q^*)^+] = \frac{nG(Q^*)}{\sqrt{n}}\\ 
\end{align*} 
Q.E.D.\\

\textbf{Problem 4 (An Investment Problem): An investor has the opportunity to make $N$ sequential investments: at time $k$ he may invest any amount $u_k\geq 0$ that does not exceed his current wealth $x_k$ (does not exceed his current wealth, $x_0$, plus his gain or minus his loss thus far). He wins his investment back and as much more with probability $p$, where $\frac{1}{2}<p<1$, and he loses his investment with probability $(1-p)$. Find the optimal investment strategy.\\}

\underline{Primitives:\\} 

$x_k$: wealth at the beginning of period $k$.\\
$u_k$: amount invested.\\
$w_k$: outcome (return).\\
$w_k=\left\{\begin{array}{ll}
2u_k & \text{w.p. }p\\
0 & \text{w.p. }1-p
\end{array}\right.$\\
where $\frac{1}{2}<p<1$.\\

\underline{Constrains:\\}

$u_k\geq 0$, \\
$u_k\leq x_k$.\\

\underline{Dynamics:\\}

$x_{k+1}=x_k-u_k+w_k$, \\

\underline{Cost:\\}

$g_N(x_N)=x_N$\\
$g_k(x_k,u_k,w_k)=0$. \\

\underline{DP algorithm:\\}

To simplify th
$J_N(x_N)=\log(x_N)$. \\
$J_k(x_k)=\underset{u_k\in \mathcal{U}_k}{\max}\mathbb{E}_{w_k}\left[J_{k+1}(x_k-u_k+w_k)\right]$.\\
We will first prove that $J_k(x_k)=A_k+\ln(x_k)$ by induction.

\begin{align}
	J_{N-1}(x_{N-1})&=\underset{u_{n-1}}{\max\ }\mathbb{E}_{w_{N-1}}\left[\log(x_{N-1}-u_{N-1}+w_{N-1})\right]=\\
	&=\underset{u_{N-1}}{\max\ }\mathbb{E}_{w_{N-1}}[\log(x_{N-1}-u_{N-1}+w_{N-1})]=\\
	&=\underset{u_{N-1}}{\max\ }\left[p\log(x_{N-1}+u_{N-1})+(1-p)\log(x_{N-1}-u_{N-1})\right].
\end{align}
Now, we derive and apply FOC to find the maximum,
\begin{align}
	\frac{p}{x_{N-1}+u_{N-1}}-\frac{1-p}{x_{N-1}-u_{N-1}}=0
\end{align}
\begin{align}
	px_{N-1}-pu_{N-1}=x_{N-1}+u_{N-1}-px_{N-1}-pu_{N-1},
\end{align}
solving we obtain $u_{N-1}=(2p-1)x_{N-1}$. Plugging into the expression of $J_{N-1}(x_{N-1})$
\begin{align}
	J_{N-1}(x_{N-1})&=
\end{align}

\end{document}
