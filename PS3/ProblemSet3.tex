\documentclass[11pt, english]{article}
\usepackage[english]{babel}
\usepackage[utf8]{inputenc}



\usepackage{geometry}
\geometry{
	a4paper,
	left=20mm,
	top=30mm,
	right=20mm
}


\usepackage{listings}
\usepackage{amsmath}
\usepackage{amsfonts}
\usepackage{amssymb}
\usepackage{amsthm} 
\usepackage{mathrsfs}
\usepackage{mathabx}
\usepackage{graphicx}
\usepackage{eurosym}
\usepackage{subfigure}
\usepackage{dsfont}
\usepackage{bbm}

%\lstdefinestyle{myCustomMatlabStyle}{
%	language=Matlab,
%	numbers=left,
%	stepnumber=1,
%	numbersep=10pt,
%	tabsize=4,
%	showspaces=false,
%	showstringspaces=false
%}

\usepackage{graphicx} 
\usepackage{fancyvrb} 
\usepackage{listings} 
\usepackage{listings}
\usepackage{amsmath}
\usepackage{amsfonts}
\usepackage{amssymb}
\usepackage{amsthm} 
\usepackage{mathrsfs}
\usepackage{mathabx}
\usepackage{graphicx}
\usepackage{eurosym}
\usepackage{subfigure}
\usepackage{dsfont}
\usepackage{bbm}
\usepackage{inputenc}

\lstdefinestyle{myCustomMatlabStyle2}{% setup listings 
	language=R,% set programming language 
	basicstyle=\small,% basic font style 
	keywordstyle=\bfseries,% keyword style 
	commentstyle=\ttfamily\itshape,% comment style 
	numbers=left,% display line numbers on the left side 
	numberstyle=\scriptsize,% use small line numbers 
	numbersep=10pt,% space between line numbers and code 
	tabsize=3,% sizes of tabs 
	showstringspaces=false,% do not replace spaces in strings by a certain character 
	captionpos=b,% positioning of the caption below 
	breaklines=true,% automatic line breaking 
	escapeinside={(*}{*)},% escaping to LaTeX 
	fancyvrb=true,% verbatim code is typset by listings 
	extendedchars=false,% prohibit extended chars (chars of codes 128--255) 
	literate={"}{{\texttt{"}}}1{<-}{{$\leftarrow$}}1{<<-}{{$\twoheadleftarrow$}}1 
	{~}{{$\sim$}}1{<=}{{$\le$}}1{>=}{{$\ge$}}1{!=}{{$\neq$}}1{^}{{$^\wedge$}}1,% item to replace, text, length of chars 
	alsoletter={.<-},% becomes a letter 
	alsoother={$},% becomes other 
	otherkeywords={!=, ~, $, *, \&, \%/\%, \%*\%, \%\%, <-, <<-, /},% other keywords 
	deletekeywords={c}% remove keywords 
}

\newcommand{\grafico}[5]{
	\begin{figure}
		[h!tbp]
		\centering
		\includegraphics[scale=#2, angle=#3]{#1}
		%\captionsetup{width=13cm}
		\caption{#4\label{#5}}
	\end{figure}
}

\newcommand{\su}[2]{\sum\limits_{#1}^{#2}}


\setlength{\parindent}{0pt}

\title{Stochastic Models and Optimization: Problem Set 3}
\author{Roger Garriga Calleja, Jos� Fernando Moreno Guti�rrez, David Rosenfeld, Katrina Walker}
\date{\today}

\begin{document}
	
\section{Q2. Inventory Pooling}
\textbf{Primitives}\\ D = demands \\ Q = quantity ordered \\ P = price \\ h =
inventory costs =  c-s \\ b = backholding costs = p-c \\ \\ First we will show that $Q_p^* = \sqrt{n}Q^* + \mu(n- \sqrt{n})$%
\begin{align*} 
G(Q) = hE[(Q-D)]^+bE[(D-Q)]^+ \\ G'(Q^*)= hP(D \leq Q^*) - b(1-P[(D\leq Q^*)]) =
0 \\ P(D \leq Q^*) = \frac{b}{h+b} \\ 
\end{align*}
Now, for the pooling we would obtain the same:
\begin{align*} 
P (\sum_{L= 1}^n D_i \leq Q_p^*) = &  \frac{b}{b+R} \\
\end{align*}
Since  this $\sum_{i = 1}^nD_i = \sqrt{n}D_i+\mu(n-\sqrt{n})$, 
\begin{align*}	
P(\sqrt{n}D_1 + \mu(n - \sqrt{n})\leq Q_p^*) = \frac{b}{b+R}  \Longleftrightarrow P(D_1 \leq
\frac{1}{\sqrt{n}}(Q_p^*- \mu(n-\sqrt{n})))= \frac{b}{b + h} \end{align*} 
which implies that $\frac{1}{\sqrt{n}}(Q_p^*- \mu(n-\sqrt{n})))$ = $Q^*$. Thus, $Q^*_P = \sqrt{n}Q^*+\mu(n-\sqrt{n})$.  Next, we will apply the hint to prove the desired result:
\begin{align*} nG(Q^*) &=
n[hE[(Q^*-D)^+]+bE[(D-Q^*)^+]]\\ G(Q^*_p) & =
[hE[(Q^*_p-\sum_{i=1}^nD_i)^+]+bE[(\sum_{i=1}^nD_i-Q^*_p)^+]]\\ 
\end{align*} 
Since $\sum_{i = 1}^nD_i = \sqrt{n}D_i+\mu(n-\sqrt{n})$ and $Q^*_P = \sqrt{n}Q^*+\mu(n-\sqrt{n})$, we can show that:\\
\begin{align*} 
G(Q^*_P) & =  [hE[(\sqrt{n}Q^*+\mu(n-\sqrt{n})-\sqrt{n}D_i-\mu(n-\sqrt{n}))^+] \\
+& b E[\sqrt{n}D_i+\mu(n-\sqrt{n})-\sqrt{n}Q^*+\mu(n-\sqrt{n})^+]]\\
& = \sqrt{n}hE[(Q^*-D_i)^+] + \sqrt{n} \ b E[(D_i-Q^*)^+] = \frac{nG(Q^*)}{\sqrt{n}}\\ 
\end{align*} 
Q.E.D.\\

\textbf{Problem 4 (An Investment Problem): An investor has the opportunity to make $N$ sequential investments: at time $k$ he may invest any amount $u_k\geq 0$ that does not exceed his current wealth $x_k$ (does not exceed his current wealth, $x_0$, plus his gain or minus his loss thus far). He wins his investment back and as much more with probability $p$, where $\frac{1}{2}<p<1$, and he loses his investment with probability $(1-p)$. Find the optimal investment strategy.\\}

\underline{Primitives:\\} 

$x_k$: wealth at the beginning of period $k$.\\
$u_k$: amount invested.\\
$w_k$: outcome (return).\\
$w_k=\left\{\begin{array}{ll}
2u_k & \text{w.p. }p\\
0 & \text{w.p. }1-p
\end{array}\right.$\\
where $\frac{1}{2}<p<1$.\\

\underline{Constrains:\\}

$u_k\geq 0$, \\
$u_k\leq x_k$.\\

\underline{Dynamics:\\}

$x_{k+1}=x_k-u_k+w_k$, \\

\underline{Cost:\\}

$g_N(x_N)=x_N$\\
$g_k(x_k,u_k,w_k)=0$. \\

\underline{DP algorithm:\\}

To simplify th
$J_N(x_N)=\log(x_N)$. \\
$J_k(x_k)=\underset{u_k\in \mathcal{U}_k}{\max}\mathbb{E}_{w_k}\left[J_{k+1}(x_k-u_k+w_k)\right]$.\\
We will first prove that $J_k(x_k)=A_k+\ln(x_k)$ by induction.

\begin{align}
	J_{N-1}(x_{N-1})&=\underset{u_{n-1}}{\max\ }\mathbb{E}_{w_{N-1}}\left[\log(x_{N-1}-u_{N-1}+w_{N-1})\right]=\\
	&=\underset{u_{N-1}}{\max\ }\mathbb{E}_{w_{N-1}}[\log(x_{N-1}-u_{N-1}+w_{N-1})]=\\
	&=\underset{u_{N-1}}{\max\ }\left[p\log(x_{N-1}+u_{N-1})+(1-p)\log(x_{N-1}-u_{N-1})\right].
\end{align}
Now, we derive and apply FOC to find the maximum,
\begin{align}
	\frac{p}{x_{N-1}+u_{N-1}}-\frac{1-p}{x_{N-1}-u_{N-1}}=0
\end{align}
\begin{align}
	px_{N-1}-pu_{N-1}=x_{N-1}+u_{N-1}-px_{N-1}-pu_{N-1},
\end{align}
solving we obtain $u_{N-1}=(2p-1)x_{N-1}$. Plugging into the expression of $J_{N-1}(x_{N-1})$
\begin{align}
	J_{N-1}(x_{N-1})&=
\end{align}

\end{document}