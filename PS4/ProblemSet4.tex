\documentclass[11pt, english]{article}
\usepackage[english]{babel}
\usepackage[utf8]{inputenc}



\usepackage{geometry}
\geometry{
	a4paper,
	left=20mm,
	top=30mm,
	right=20mm
}

\usepackage{listings}
\usepackage{amsmath}
\usepackage{amsfonts}
\usepackage{amssymb}
\usepackage{amsthm} 
\usepackage{mathrsfs}
\usepackage{mathabx}
\usepackage{graphicx}
\usepackage{eurosym}
\usepackage{subfigure}
\usepackage{dsfont}
\usepackage{bbm}

%\lstdefinestyle{myCustomMatlabStyle}{
%	language=Matlab,
%	numbers=left,
%	stepnumber=1,
%	numbersep=10pt,
%	tabsize=4,
%	showspaces=false,
%	showstringspaces=false
%}

\usepackage{graphicx} 
\usepackage{fancyvrb} 
\usepackage{listings} 
\usepackage{listings}
\usepackage{amsmath}
\usepackage{amsfonts}
\usepackage{amssymb}
\usepackage{amsthm} 
\usepackage{mathrsfs}
\usepackage{mathabx}
\usepackage{graphicx}
\usepackage{eurosym}
\usepackage{subfigure}
\usepackage{dsfont}
\usepackage{bbm}
\usepackage{inputenc}

\lstdefinestyle{myCustomMatlabStyle2}{% setup listings 
	language=R,% set programming language 
	basicstyle=\small,% basic font style 
	keywordstyle=\bfseries,% keyword style 
	commentstyle=\ttfamily\itshape,% comment style 
	numbers=left,% display line numbers on the left side 
	numberstyle=\scriptsize,% use small line numbers 
	numbersep=10pt,% space between line numbers and code 
	tabsize=3,% sizes of tabs 
	showstringspaces=false,% do not replace spaces in strings by a certain character 
	captionpos=b,% positioning of the caption below 
	breaklines=true,% automatic line breaking 
	escapeinside={(*}{*)},% escaping to LaTeX 
	fancyvrb=true,% verbatim code is typset by listings 
	extendedchars=false,% prohibit extended chars (chars of codes 128--255) 
	literate={"}{{\texttt{"}}}1{<-}{{$\leftarrow$}}1{<<-}{{$\twoheadleftarrow$}}1 
	{~}{{$\sim$}}1{<=}{{$\le$}}1{>=}{{$\ge$}}1{!=}{{$\neq$}}1{^}{{$^\wedge$}}1,% item to replace, text, length of chars 
	alsoletter={.<-},% becomes a letter 
	alsoother={$},% becomes other 
	otherkeywords={!=, ~, $, *, \&, \%/\%, \%*\%, \%\%, <-, <<-, /},% other keywords 
	deletekeywords={c}% remove keywords 
}

\newcommand{\grafico}[5]{
	\begin{figure}
		[h!tbp]
		\centering
		\includegraphics[scale=#2, angle=#3]{#1}
		%\captionsetup{width=13cm}
		\caption{#4\label{#5}}
	\end{figure}
}

\newcommand{\su}[2]{\sum\limits_{#1}^{#2}}

\setlength{\parindent}{0pt}
\begin{document}
\title{Stochastic Models and Optimization: Problem Set 4}
\author{Roger Garriga Calleja, José Fernando Moreno Gutiérrez, David Rosenfeld, Katrina Walker}
\date{\today}
\maketitle
\section*{Q1}
\section*{Q2}
\section*{Q3}
\textbf{Asset selling w/offer estimation}\\
\\
\underline{Primitives}\\
\begin{itemize}
\item $x_k$ current offer.
\item  $w_0, w_1,.....w_{n-1}$ of iid offers with unknown distribution
\item an underlying distribution of the offers $w$ (i.e. the hidden state $x_k$) $F_1$ or $F_2$, thus $y_k = y^1$ if true distribution is $F_1$ and $y^2$ if the true distribution is $F_2$
\item 
constraints (if seller sells ($u_1$) or not ($u_2$)):
$\left \{\begin{tabular}{lll}
	{$u^1, u^2$} \ \textup{if} \ $x_k$ $\neq$ T\\
	0,  otherwise
\end{tabular} \right\}\\ 
$\item rewards:
$g_n(x_N) =$
$\left \{
\begin{tabular}{lll}
 $x_N$, \textup{if} \ $x_N$ $\neq$ T\\
0,  otherwise
\end{tabular}\right \}\\
g_k(x_k, u_k, w_k)$ = 
$ \left \{ 
\begin{tabular}{lll}
$(1 + r)^{N-k}x_k$, \textup{if} \ $x_k$ $\neq$ T and if $u_k$ = $u^1$\\
	0,  otherwise 
\end{tabular}
\right \}$
\item $q$ = prior belief that $F_1$ is true 
\item $q_{k+1} = \frac{\mathds{P}\{y_k = y^1|w_0,\dots,w_{k}\}}{\mathds{P}(w_1 = w_1)}$
= $\frac{q_kF_1(w_k)}{q_kF_1(w_k) + (1-q)F_2(w_k)}$
\end{itemize}
Now, we can apply the DP algorithm to find an optimal asset selling policy\\ 
$J_{N-1}(P_{N-1}) = $
$ \left \{ 
\begin{tabular}{lll}
$(P_{N-1}\mathds{E}_{F_1}[w_{N-1}] + (1-P_{N-1})\mathds{E}_{F_2}[w_{N-1}])(1 + r)^{N-k}$\\
0,  otherwise 
\end{tabular}\\
\right \}\textup{if} \ x_{N-1} \neq T\\$
\\
$ J_k(x_k) = $
$ \left \{
\begin{tabular}{lll}
$\textup{max}(P_{k}\mathds{E}_{F_1}[w_{k}] + (1-P_{k})\mathds{E}_{F_2}[w_{k}])(1 + r)^{k}, \mathds{E}[J_{k+1}(w_k)]$\\
	0,  otherwise 
\end{tabular}
\right \} \textup{if} \ x_k \neq$ T \\
Thus, the threshold for selling an asset will be:
$P_k\mathds{E}_{F_1}(w_k) + (1 - P_k)\mathds{E}_{F_2}(w_k) \geq \frac{\mathds{E}[J_{k+1}(w_k)]}{(1 +r)^{n-k}}$ \\
And the optimal asset selling policy:
$\mu^*(x_k) =$
$ \left \{
\begin{tabular}{lll}
$u^*, \frac{\mathds{E}[J_{k+1}(w_k)]}{(1 +r)^{n-k}}$\\
$u^2$, otherwise
\end{tabular}
\right \}\\$

\section*{Q4}



This problem is basically the same as the inventory management considering the demand as a random variable following an unknown distribution. It is a case with imperfect state information, in which the distribution of demand will be either $F_1$ or $F_2$. The probability that the demand follows $F_1$ is updated at each period $k$ after observing the realization of the demand. That will effect the way the expectation of the demand is computed.\\

\underline{Primitives:\\}

$x_k$: items in the inventory at period $k$.\\
$u_k$: quantity ordered at period $k$.\\
$w_k$: demand during period $k$. $w_k$ are iid with probability distribution either $F_1$ or $F_2$.\\
$q_k$: probability that $w_k$ follows distribution $F_1$.\\
$q_0=q$: a priori probability that demand follows the distribution $F_1$.\\

\underline{Dynamics:\\}

$x_{k+1}=x_k+u_k-w_k$\\
$q_{k+1}=\frac{q_kf_1(w_k)}{q_kf_1(w_k)+(1-q_k)f_2(w_k)}$, where $f_i(w)$ is the pdf of the distribution $F_i$.\\

\underline{Cost:\\}

$g_N(x_N)=0$.\\
$g_k(x_k,u_k,w_k)=cu_k+h\max\{0,w_k-x_k-u_k\}+p\max\{0,x_k+u_k-w_k\}$, where $c,h,p$ are positive and $p>c$.\\

\underline{DP algorithm:}\\

$J_N(x_N)=0$\\
$J_k(x_k)=\underset{u_k\geq 0}{\min}\mathbb{E}\left[cu_k+h\max\{0,w_k-x_k-u_k\}+p\max\{0,x_k+u_k-w_k\}+J_{k+1}(x_{k+1})\right]$\\

In order to solve it we can introduce the variable $y_k=x_k+u_k$, and the we have \\
$J_k(y_k)=\underset{u_k\geq x_k}{\min}G_k(y_k)-cx_k$, where $$G_k(y_k)=cy+h\mathbb{E}[\max\{0,w_k-y_k\}]+p\mathbb{E}[\max\{0,y_k-w_k\}]+\mathbb{E}[J_{k+1}(y_k-w_k)].$$
Now, since $w_k$ is drawn from $F_1$ with probability $q_k$ and from $F_2$ with probability $F_2$ we can apply the law of total probabilities, leading to\\
\begin{align*}
G(y_k)=cy_k+q_k(h\mathbb{E}_{w_k|w\sim F_1}[\max\{0,w_k-y_k\}]+p\mathbb{E}_{w_k|w\sim F_1}[\max\{0,y_k-w_k\}]+\mathbb{E}_{w_k|w\sim F_1}[J_{k+1}(y_k-w_k)])+\\
+(1-q_k)(h\mathbb{E}_{w_k|w\sim F_2}[\max\{0,w_k-y_k\}]+p\mathbb{E}_{w_k|w\sim F_2}[\max\{0,y_k-w_k\}]+\mathbb{E}_{w_k|w\sim F_2}[J_{k+1}(y_k-w_k)]).
\end{align*}
We saw in class that $cy_k+h\mathbb{E}_{w_k|w\sim F_i}[\max\{0,w_k-y_k\}]+p\mathbb{E}_{w_k|w\sim F_i}[\max\{0,y_k-w_k\}]+\mathbb{E}_{w_k|w\sim F_i}[J_{k+1}(y_k-w_k)]$ is convex, since we have a sum of convex, our $G(y_k)$ will also be convex. So, there exists a $S_k$ that will represent the optimal stock we seek at period $k$. However, $S_k$ could be smaller than $x_k$, so it would not be reachable (in which case we would not by stock). Then, the policy will be
\begin{equation*}
\mu_k^*(x_k)=\left\{\begin{array}{ll}
S_k-x_k & \text{if } S_k>x_k\\
0 & \text{otherwise.}
\end{array}\right.
\end{equation*}  

\end{document}
