\documentclass[11pt, english]{article}
\usepackage[english]{babel}
\usepackage[utf8]{inputenc}

\usepackage{geometry}
\geometry{
	a4paper,
	left=20mm,
	top=30mm,
	right=20mm
}

\usepackage{listings}
\usepackage{amsmath}
\usepackage{amsfonts}
\usepackage{amssymb}
\usepackage{amsthm} 
\usepackage{mathrsfs}
\usepackage{mathabx}
\usepackage{graphicx}
\usepackage{eurosym}
\usepackage{subfigure}
\usepackage{dsfont}
\usepackage{bbm}
%\lstdefinestyle{myCustomMatlabStyle}{
%	language=Matlab,
%	numbers=left,
%	stepnumber=1,
%	numbersep=10pt,
%	tabsize=4,
%	showspaces=false,
%	showstringspaces=false
%}

\usepackage{graphicx} 
\usepackage{fancyvrb} 
\usepackage{listings} 
\usepackage{listings}
\usepackage{amsmath}
\usepackage{amsfonts}
\usepackage{amssymb}
\usepackage{amsthm} 
\usepackage{mathrsfs}
\usepackage{mathabx}
\usepackage{graphicx}
\usepackage{eurosym}
\usepackage{subfigure}
\usepackage{dsfont}
\usepackage{bbm}
\usepackage{inputenc}

% Default fixed font does not support bold face
\DeclareFixedFont{\ttb}{T1}{txtt}{bx}{n}{12} % for bold
\DeclareFixedFont{\ttm}{T1}{txtt}{m}{n}{12}  % for normal

% Custom colors
\usepackage{color}
\definecolor{deepblue}{rgb}{0,0,0.5}
\definecolor{deepred}{rgb}{0.6,0,0}
\definecolor{deepgreen}{rgb}{0,0.5,0}

\newcommand\pythonstyle{\lstset{
		language=Python,
		%basicstyle=\ttm,
		otherkeywords={self},             % Add keywords here
		keywordstyle=\ttb\color{deepblue},
		emph={MyClass,__init__},          % Custom highlighting
		emphstyle=\ttb\color{deepred},    % Custom highlighting style
		stringstyle=\color{deepgreen},
		frame=tb,                         % Any extra options here
		showstringspaces=false            % 
		breaklines=true,% automatic line breaking
		basicstyle=\small,% basic font style 
		numbers=left,% display line numbers on the left side 
		numberstyle=\scriptsize,% use small line numbers 
		numbersep=10pt,% space between line numbers and code 
		tabsize=3,% sizes of tabs 
		showstringspaces=false,% do not replace spaces in strings by a certain character 
		captionpos=b,% positioning of the caption below 
		breaklines=true,% automatic line breaking 
}}


% Python environment
\lstnewenvironment{python}[1][]
{
	\pythonstyle
	\lstset{#1}
}
{}

% Python for external files
\newcommand\pythonexternal[2][]{{
		\pythonstyle
		\lstinputlisting[#1]{#2}}}

% Python for inline
\newcommand\pythoninline[1]{{\pythonstyle\lstinline!#1!}}


\lstdefinestyle{myCustomMatlabStyle2}{% setup listings 
	language=R,% set programming language 
	basicstyle=\small,% basic font style 
	keywordstyle=\bfseries,% keyword style 
	commentstyle=\ttfamily\itshape,% comment style 
	numbers=left,% display line numbers on the left side 
	numberstyle=\scriptsize,% use small line numbers 
	numbersep=10pt,% space between line numbers and code 
	tabsize=3,% sizes of tabs 
	showstringspaces=false,% do not replace spaces in strings by a certain character 
	captionpos=b,% positioning of the caption below 
	breaklines=true,% automatic line breaking 
	escapeinside={(*}{*)},% escaping to LaTeX 
	fancyvrb=true,% verbatim code is typset by listings 
	extendedchars=false,% prohibit extended chars (chars of codes 128--255) 
	literate={"}{{\texttt{"}}}1{<-}{{$\leftarrow$}}1{<<-}{{$\twoheadleftarrow$}}1 
	{~}{{$\sim$}}1{<=}{{$\le$}}1{>=}{{$\ge$}}1{!=}{{$\neq$}}1{^}{{$^\wedge$}}1,% item to replace, text, length of chars 
	alsoletter={.<-},% becomes a letter 
	alsoother={$},% becomes other 
	otherkeywords={!=, ~, $, *, \&, \%/\%, \%*\%, \%\%, <-, <<-, /},% other keywords 
	deletekeywords={c}% remove keywords 
}

\newcommand{\grafico}[5]{
	\begin{figure}
		[h!tbp]
		\centering
		\includegraphics[scale=#2, angle=#3]{#1}
		%\captionsetup{width=13cm}
		\caption{#4\label{#5}}
	\end{figure}
}

\newcommand{\su}[2]{\sum\limits_{#1}^{#2}}

\setlength{\parindent}{0pt}
\begin{document}
\title{Stochastic Models and Optimization: Problem Set 4}
\author{Roger Garriga Calleja, José Fernando Moreno Gutiérrez, David Rosenfeld, Katrina Walker}
\date{\today}
\maketitle
\section*{Q1}

We are given a linear-quadratic problem with perfect state information but with a forecast. We first set up the primitives of the system:\\
$x_k$: the state in period k\\
$u_k$: the decision variable in period k\\
$w_k$: the disturbances in period k\\
$y_k$: an accurate prediction that $w_k$ will be selected according to a particular probability distribution $P_{k|y_k}$\\
\\
We set up the dynamics of the problem with a linear system, as follows:
$$x_{k+1} = A_kx_k + B_ku_k + w_k$$
Where A and B are $n \times n$ matrices.\\
We also have a quadratic cost:
$$g_N(x_N) = x_N'Q_Nx_N$$
$$g_k(x_k) = x_k'Q_kx_k + u_k'R_ku_k$$
Where $Q_k$ and $R_k$ are $n \times n$ positive definite matrices.\\
\\
Our problem is thus to minimise:
$$E[\sum_{k=0}^{N-1}(x_k'Q_kx_k + u_k'R_ku_k) + x_N'Q_Nx_N]$$
Subject to:
$$x_{k+1} = A_kx_k + B_ku_k + w_k$$
\\
We can now set up our DP-algorithm:
$$J_N(x_N, y_N) = x_N'Q_Nx_N$$
$$J_k(x_k, y_k) = \min_{u_k \in R^n}E_{w_k}[x_k'Q_kx_k + u_k'R_ku_k + J_{k+1}(x_{k+1}, y_{k+1})]$$
\\
We can see that $J_N(x_N)$ is of the form $J(x_k, y_k) = x_k'K_kx_k + x_k'b_k(y_k) + c(y_k)$, with$Q_n = K_n$, $x_N'b_N(y_N) = 0$ and $c(y_k) = 0$.\\
We assume that this is also true at stage k+1, so that:
$$J(x_{k+1}, y_{k+1}) = x_{k+1}'K_{k+1}x_{k+1} + x_{k+1}'b_{k+1}(y_{k+1}) + c(y_{k+1})$$
Where $b_{k+1}(y_{k+1})$ is an n-dimensional vector and $c(y_{k+1})$ is a scalar.
Using this, we can compute $J_k(x_k, y_k)$:
\begin{align*}
	J_k(x_k, y_k) &=  \min_{u_k \in R^n}E_{w_k}[x_k'Q_kx_k + u_k'R_ku_k + J_{k+1}(x_{k+1}, y_{k+1})]=\\
	& = \min_{u_k \in R^n}E_{w_k}[x_k'Q_kx_k + u_k'R_ku_k + x_{k+1}'K_{k+1}x_{k+1} + x_{k+1}'b_{k+1}(y_{k+1}) + c(y_{k+1})\\
	&= \min_{u_k \in R^n}E_{w_k}[x_k'Q_kx_k + u_k'R_ku_k + (A_kx_k + B_ku_k + w_k)'K_{k+1}(A_kx_k + B_ku_k + w_k) +\\
	&+ (A_kx_k + B_ku_k + w_k)'b_{k+1}(y_{k+1}) + c(y_{k+1})]= x_k'(Q_k + A_k'K_{k+1}A_k)x_k + E[w_k'K_{k+1}w_k] +\\
	&+ \min_{u_k \in R^n}[u_k'(R_k + B_k'K_{k+1}B_k)u_k + 2x_k'A_kK_{k+1}B_ku_k + 2u_kB_kK_{k+1}E[w_k|y_k] +\\
	&+ u_k'B_k'b_{k+1}(y_{k+1})] + x_k'A_k'K_{k+1}E[w_k|y_k] + x_k'A_k'b_{k+1}(y_{k+1}) + E[w_k|y_k]'b_{k+1}(y_{k+1})
\end{align*}


We then find our optimal decision by taking the derivative of $J_k(x_k, y_k)$ with respect to $u_k$ and set it equal to zero in order to solve for our optimal solution $u*$:
$$2(R_k + B_k'K_{k+1}B_k)u_k^* + 2B_k'K_{k+1}A_kx_k + 2B_kK_{k+1}B_k + 2B_kK_{k+1}E[w_k|y_k] + B_k'b_{k+1}(y_{k+1}) = 0$$
$$u_k^* = (R_k + B_k'K_{k+1}B_k)^{-1}B_k'K_{k+1}(A_kx_k + E[w_k|y_k]) + \alpha_k$$
Where $\alpha_k = B_k'b_{k+1}(y_{k+1})$
\newpage
\section*{Q2}
\textbf{(i)}\\

Controllability is related to the stability of the system. It allows that the eigenvalues of $A + BL$ could have magnitude less than 1, and then the closed loop system is stable. It also is an important factor for determining the convergence rate of the system. Observability is an important condition for making K merely positive definite. Thus, controllability and observability allow K to satisfy the Algebraic Riccati Equation and guarantee uniqueness of the optimal control.\\

\textbf{(ii, iii, iv, v)}\\

\textbf{Python Code:}\\

\pythonexternal{code.py}

\begin{center}
	\begin{figure}[ht!]
		\centering
		\includegraphics[height=7cm,width=8cm, page=1]{figures}
		\includegraphics[height=7cm,width=8cm, page=2]{figures}
		\includegraphics[height=7cm,width=8cm, page=3]{figures}
		\includegraphics[height=7cm,width=8cm, page=4]{figures}
	\end{figure}
\end{center}

\newpage
\section*{Q3}
\textbf{Asset selling w/offer estimation}\\
\\
\underline{Primitives}\\
\begin{itemize}
\item $x_k$ current offer.
\item  $\{w_k\}$ iid offers with an unknown underlying distribution $F_1$ or $F_2$.
\item constraints: selling ($u_1$) or not selling ($u_2$)):
$\left \{\begin{tabular}{lll}
	\{$u^1, u^2$\} \ \textup{if} \ $x_k$ $\neq$ T\\
	0,  otherwise
\end{tabular} \right.\\ 
$\item rewards:
$g_n(x_N) =$
$\left \{
\begin{tabular}{lll}
 $x_N$, \textup{if} \ $x_N$ $\neq$ T\\
0,  otherwise
\end{tabular}\right.\\
g_k(x_k, u_k, w_k)$ = 
$ \left \{ 
\begin{tabular}{lll}
$(1 + r)^{N-k}x_k$, \textup{if} \ $x_k$ $\neq$ T and if $u_k$ = $u^1$\\
	0,  otherwise 
\end{tabular}
\right.$
\item $q$ = prior belief that $F_1$ is true 
\item $q_{k+1} = \frac{\mathds{P}\{y_k = y^1|w_0,\dots,w_{k}\}}{\mathds{P}(w_1 = w_1)}$
= $\frac{q_kF_1(w_k)}{q_kF_1(w_k) + (1-q)F_2(w_k)}$
\end{itemize}
Now, we can apply the DP algorithm to find an optimal asset selling policy\\ \\
$J_{N-1}(x_{N-1}) = $
$ \left \{ 
\begin{tabular}{lll}
$\max\{(1 + r)x_k,(q_{N-1}\mathds{E}_{F_1}[J_N(w_{N-1})] + (1-q_{N-1})\mathds{E}_{F_2}[J_{N}(w_{N-1})])$ if $x_{N-1}\neq T$\\
0,  otherwise 
\end{tabular}\\
\right.$
\\\\
$ J_k(x_k) = $
$ \left\{
\begin{tabular}{lll}
$\textup{max}(q_{k}\mathds{E}_{F_1}[J_{k+1}(w_{k})] + (1-q_{k})\mathds{E}_{F_2}[J_{k+1}(w_{k})]),(1 + r)^{N-k}x_k$ if $x_k\neq T$\\
	0,  otherwise 
\end{tabular}
\right.$\\\\
Thus, the threshold for selling an asset will be:
$(1 + r)^{N-k}x_k\geq q_{k}\mathds{E}_{F_1}[J_{k+1}(w_{k})] + (1-q_{k})\mathds{E}_{F_2}[J_{k+1}(w_{k})]$ \\\\
So, the optimal asset selling policy will be:
$\mu^*(x_k) =$
$ \left \{
\begin{tabular}{lll}
$u^1, x_{k+1}\geq \frac{q_{k}\mathds{E}_{F_1}[J_{k+1}(w_{k})] + (1-q_{k})\mathds{E}_{F_2}[J_{k+1}(w_{k})]}{(1 + r)^{N-k}}$\\
$u^2$, otherwise
\end{tabular}
\right.$
\newpage
\section*{Q4}

This problem is basically the same as the inventory management considering the demand as a random variable following an unknown distribution. It is a case with imperfect state information, in which the distribution of demand will be either $F_1$ or $F_2$. The probability that the demand follows $F_1$ is updated at each period $k$ after observing the realization of the demand. That will effect the way the expectation of the demand is computed.\\

\underline{Primitives:\\}

$x_k$: items in the inventory at period $k$.\\
$u_k$: quantity ordered at period $k$.\\
$w_k$: demand during period $k$. $w_k$ are iid with probability distribution either $F_1$ or $F_2$.\\
$q_k$: probability that $w_k$ follows distribution $F_1$.\\
$q_0=q$: a priori probability that demand follows the distribution $F_1$.\\

\underline{Dynamics:\\}

$x_{k+1}=x_k+u_k-w_k$\\
$q_{k+1}=\frac{q_kf_1(w_k)}{q_kf_1(w_k)+(1-q_k)f_2(w_k)}$, where $f_i(w)$ is the pdf of the distribution $F_i$.\\

\underline{Cost:\\}

$g_N(x_N)=0$.\\
$g_k(x_k,u_k,w_k)=cu_k+h\max\{0,w_k-x_k-u_k\}+p\max\{0,x_k+u_k-w_k\}$, where $c,h,p$ are positive and $p>c$.\\

\underline{DP algorithm:}\\

$J_N(x_N)=0$\\
$J_k(x_k)=\underset{u_k\geq 0}{\min}\mathbb{E}\left[cu_k+h\max\{0,w_k-x_k-u_k\}+p\max\{0,x_k+u_k-w_k\}+J_{k+1}(x_{k+1})\right]$\\

In order to solve it we can introduce the variable $y_k=x_k+u_k$, and the we have \\
$J_k(y_k)=\underset{u_k\geq x_k}{\min}G_k(y_k)-cx_k$, where $$G_k(y_k)=cy+h\mathbb{E}[\max\{0,w_k-y_k\}]+p\mathbb{E}[\max\{0,y_k-w_k\}]+\mathbb{E}[J_{k+1}(y_k-w_k)].$$
Now, since $w_k$ is drawn from $F_1$ with probability $q_k$ and from $F_2$ with probability $F_2$ we can apply the law of total probabilities, leading to\\
\begin{align*}
G(y_k)=cy_k+q_k(h\mathbb{E}_{w_k|w\sim F_1}[\max\{0,w_k-y_k\}]+p\mathbb{E}_{w_k|w\sim F_1}[\max\{0,y_k-w_k\}]+\mathbb{E}_{w_k|w\sim F_1}[J_{k+1}(y_k-w_k)])+\\
+(1-q_k)(h\mathbb{E}_{w_k|w\sim F_2}[\max\{0,w_k-y_k\}]+p\mathbb{E}_{w_k|w\sim F_2}[\max\{0,y_k-w_k\}]+\mathbb{E}_{w_k|w\sim F_2}[J_{k+1}(y_k-w_k)]).
\end{align*}
We saw in class that $cy_k+h\mathbb{E}_{w_k|w\sim F_i}[\max\{0,w_k-y_k\}]+p\mathbb{E}_{w_k|w\sim F_i}[\max\{0,y_k-w_k\}]+\mathbb{E}_{w_k|w\sim F_i}[J_{k+1}(y_k-w_k)]$ is convex, since we have a sum of convex, our $G(y_k)$ will also be convex. So, there exists a $S_k$ that will represent the optimal stock we seek at period $k$. However, $S_k$ could be smaller than $x_k$, so it would not be reachable (in which case we would not by stock). Then, the policy will be
\begin{equation*}
\mu_k^*(x_k)=\left\{\begin{array}{ll}
S_k-x_k & \text{if } S_k>x_k\\
0 & \text{otherwise.}
\end{array}\right.
\end{equation*}  
\newpage
\section*{Q5}

(a) To find the policy that will minimize the cost function in the worst case. We can write a DP-like algorithm to solve this problem as\\
$J_N(x_N)=g_N(x_N)$\\
$J_k(x_k)=\underset{\mu_k\in U_k}{\min\ }\underset{w_k\in W_k(x_k,\mu_k(x_k))}{\max\ }[g_k(x_k,\mu_k(x_k),w_k)+J_{k+1}(x_{k+1})]$, where $x_{k+1}=f_k(x_k,\mu_k(x_k),w_k)$ $\forall k=0,\dots, N-1$.\\

(b) In this problem the state $x_k$ needs to belong to a certain set $X_k$ at each state $k$, so we have to choose a policy that will assure that this will happen. We write the dynamics in a general form like \\
$x_{k+1}=f(x_k,\mu_k(x_k))+h_k(w_k)$.\\
We consider a cost of the policy $g(\mu_k(x_k))$ and the cost of being outside $X_k$ infinite to ensure we stay whenever possible there.\\

\underline{DP algorithm:}\\

$J_N(x_N)=\left\{\begin{array}{ll}
0 & \text{if }x_N\in X_N\\
\infty & \text{otherwise}
\end{array}\right.$\\
$J_{k+1}=\underset{\mu_k(x_k)\in U_k}{\min\ }\underset{w_k\in W_k(x_k,\mu_k(x_k))}{\max\ }\left\{\begin{array}{ll}
g_k(\mu(x_k))+J_{k+1}(x_{k+1}) & \text{if } x_k\in X_k\\
\infty & \text{otherwise}
\end{array}\right.$,\\\\ where $x_{k+1}=f_k(x_k,\mu_k(x_k))+h_k(w_k)$. Then, we will need to compute recursively the $\bar{X}_k$ that allow to continue in the sets $X_{k+1},\dots,X_N$. In order to do so, $x_k$ will need to be in a certain set $Y_{k+1}$ such that for any $w_k\in W_k$, $f(x_k,\mu_k(x_k))+h(w_k)$ belong to $\bar{X}_{k+1}$. Initializing $\bar{X}_N=X_N$, $\forall k=1,\dots,N-1$, the recursion would be the next:\\

$Y_{k+1}=\{z\in \mathcal{X}: z+h_k(w_k)\in\bar{X}_{k+1},\forall w_k\in W_k(x_k,\mu_k(x_k))\}$\\

$\bar{X}_k=\{x\in X_k: \exists \mu_k(x)\in U_k\text{ such that }  f_k(x,\mu_k(x_k))\in Y_{k+1}\}$




\end{document}
