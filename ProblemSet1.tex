\documentclass[11pt, english]{article}
\usepackage[english]{babel}
\usepackage[utf8]{inputenc}



\usepackage{geometry}
 \geometry{
 a4paper,
 left=20mm,
 top=30mm,
 right=20mm
 }


\usepackage{listings}
\usepackage{amsmath}
\usepackage{amsfonts}
\usepackage{amssymb}
\usepackage{amsthm} 
\usepackage{mathrsfs}
\usepackage{mathabx}
\usepackage{graphicx}
\usepackage{eurosym}
\usepackage{subfigure}
\usepackage{dsfont}
\usepackage{bbm}

\newcommand{\grafico}[5]{
\begin{figure}
[h!tbp]
\centering
\includegraphics[scale=#2, angle=#3]{#1}
%\captionsetup{width=13cm}
\caption{#4\label{#5}}
\end{figure}
}

\newcommand{\su}[2]{\sum\limits_{#1}^{#2}}


\setlength{\parindent}{0pt}

\title{Stochastic Models and Optimization: Problem Set 1}
\author{Roger Garriga Calleja, José Fernando Moreno Gutiérrez, David Rosenfeld, Katrina Walker}
\date{\today}

\begin{document}
\maketitle

\textbf{Problem 3 (Multiplicative cost):}\textit{ In the framework of the basic problem, consider the case where the cost has the multiplicative form}
$$\mathbb{E}_{\{w_k\}}\left\{g_N(x_N)\cdot g_{N-1}(x_{N-1},u_{N-1},w_{N-1})\cdots g_0(x_0,u_0,w_0) \right\}$$
\textit{Develop a DP-like algorithm for this problem assuming that $g_k(x_k,u_k,w_k)>0$, for all $x_k,u_k,w_k$ and $k$.}\\
\underline{Primitives:}\\
State: The state of the system at the beginning of the period $k$ is $x_k$.\\
Control: The control or decision at period $k$ is $u_k$.\\
Uncertainty: The uncertainty at period $k$ is $w_k$.\\
Dynamics: The dynamics of the problem at period $k$ is $x_{k+1}=f_k(x_k,u_k,w_k)$, for a certain $f_k$.\\
Cost: The cost at period $k$ is $f_k(x_k,u_k,w_k)$ and has a multiplicative form (so the cost from the period 0 to the period $k\neq N$ will be $\prod\limits_{i=0}^k g_i(x_i,u_i,w_i)$ and $g_N(x_N)\prod\limits_{i=0}^{k-1} g_i(x_i,u_i,w_i)$ if $k=N$).
 
As in the DP problem, we can take $J_k(x_k)$ as the cost-to-go of the $N-k$ element. Then, 
$$J_N(x_N)=g(x_N).$$
And from there on considering $U_k$ the set of the possible decisions,
$$J_{k}(x_k)=\underset{u_k\in U_k}{\min}\mathbb{E}_{w_k}\left\{g_k(x_k,u_k,w_k)J_{k+1}(f_k(x_k,u_k,w_k))\right\}.$$
Let's prove that this makes sense, meaning that if the minimum of the expected cost-to-go in the $(k+1)$th state is $J_{k+1}(x_{k+1})$ then the minimum of the expected cost-to-go in the $k$th state will be $J_k(x_k)$ as we have defined it. To do so, we will use induction. \\

Let us call $J^*(x_k)$ the real minimum of the expected cost-to-go in the $k$th state. It is clear that $J_N^*(x_N)=J_N(x_N)=g_N(x_N)$ (because it does not depend on any $w$). Now, assume that $J_{k+1}^*(x_{k+1})=J_{k+1}(x_{k+1})$. Then, considering $J_k^*(x_k)$,
$$J_k^*(x_k)=\underset{u_{k}\dots, u_{N-1}}{\min}\mathbb{E}_{w_{k},\dots,w_{N-1}}\left\{g_k(x_k,u_k,w_k)\prod\limits_{i=k+1}^{N-1}
\left(g_i(x_i,u_i,w_i)\right)g_N(x_N)\right\}.$$
Since $g_k(x_k,u_k,w_k)$ only depends on period $k$ we can put it out of the expectation and the minimum over the subsequent variables ($u_{k+1},\dots,u_{N-1}$ and $w_{k+1},\dots,w_{N-1}$). So we get

\begin{align*}
	J_k^*(x_k) & =\underset{u_k}{\text{min }}\mathbb{E}_{w_k}\left\{g_k(x_k,u_k,w_k)\underset{u_{k+1},\dots,u_{N-1}}{\min}\mathbb{E}_{w_{k+1},\dots,w_{N-1}}\left\{\prod\limits_{i=k+1}^{N-1}\left(g_i(x_i,u_i,w_i)\right)g_N(x_N)\right\} \right\}=\\
	& =\underset{u_k}{\text{min }}\mathbb{E}_{w_k}\left\{g_k(x_k,u_k,w_k)J_{k+1}^*(x_{k+1})\right\},
\end{align*}
 
since $J_{k+1}(x_{k+1})=J_{k+1}^*(x_{k+1})=\underset{u_{k+1},\dots,u_{N-1}}{\min}\mathbb{E}_{w_{k+1},\dots,w_{N-1}}\left\{\prod\limits_{i=k+1}^{N-1}\left(g_i(x_i,u_i,w_i)\right)g_N(x_N)\right\}$ by hypothesis of induction,
$$J_k^*(x_k)=\underset{u_k}{\text{min }}\mathbb{E}_{w_k}\left\{g_k(x_k,u_k,w_k)J_{k+1}(x_{k+1})\right\}.$$
Q.E.D.

\textbg{Problem 4 - Knapsack Problem:} \textit{Assume that we have a vessel whose maximum weight capacity is $z$ and whose cargo is to consist of different quantities of $N$ different items. Let $v_i$ denote the value of the $i^{th}$ type of item, $w_i$ the weight of the $i^{th}$ type of item, and $x_i$ the number of items of type $i$ that are loaded in the vessel. The problem is to find the most valuable cargo, i.e., to maximize $\sum_{i = 1}^Nx_iv_i$ subject to the constraint $\sum_{i = 1}^Nx_iv_i \leq z$ and $x_i \in \mathbb{N}$. Reformulate the problem within the DP framework.}

We can define the primitives as follow:\\
$x_k$: the total weight on the vessel before adding the $k^{th}$ item.\\
$u_k$: the quantity of item k added to the vessel.\\
$w_k$: the weight of a unit of item k.\\
$z$: the maximum weight capacity of the vessel.\\
$v_k$: the value of one unit of item k.
\\
The dynamics are as follows:\\
$x_{k+1} = x_k + u_k.w_k$\\
$U_k = {u_k: x_k + u_k.w_k \leq z}$\\
$x_i \in N$\\
$g_k(x_k, u_k) = u_k.v_k$
$g_N(x_N) =[\frac{z - x_N}{w_n}].v_N$ (where $[\frac{z - x_N}{w_n}] \in N$)\\
\\
We may now set up our DP algorithm:\\
$J_N(x_N) = [\frac{z - x_N}{w_n}].v_N$
$J_k(x_k) = max{g_k(x_k, u_k) + J_{k+1}(x_{k+1}$\\
$= max{u_{N-1}.v_{N-1} + J_{k+1}(x_{k+1}$


\textbf{Problem 5 (Traveling Repairman Problem):}\textit{ A repairman must service N sites, which are located along a line and are sequentially numbered 1, 2, . . . , N. The repairman starts at a given site s with 1 $<$ s $<$ N, and is constrained to service only sites that are adjacent to the ones serviced so far, i.e. if he has already serviced sites i, i+1,...j, then he may service only site i-1 (assuming 1 \textless i) or that site j+1 (assuming j \textless N). There is a waiting cost of $c_i$ for each time period that site i has remained unserviced and there is a travel cost $t_{ij}$ for servicing site j right after site i. Reformulate the problem within the DP framework.}\\
\\
\underline{Primitives:}\\
S: the origin at period k\\
T: the destination\\ 
$\alpha_{ij}$: $t_{ij} + \sum_{j = k+1}^N c_j$ \\
$d_i$ : path of minimum total cost from s to i \\
$P_i$ : parent of node \\
Open: set of nodes whose labels may need correction \\
Upper: lowest costs from s - t \\
\\ \underline{Initialization:}\\
\begin{align*}
d_S &= \sum_{j = 1}^N c_j \\ 
d_i = \inf \\ 
\forall \ j \neq S \\
\textup{Open} &= {S} \\
\textup{Upper} &= \infty \\
\end{align*}
\underline{Algorithm:}\\
\begin{align*}
1) \ \textup{Remove} \ i \in \textup{Open} \\
\forall \ j \ \textup{child of i execute} \ (2) \\
2) \ \textup{if}\ d_i + \alpha_{ij} \textless \ \textup{min}\ \left\{d_j, \textup{Upper}\right\} \\
\textup{then} \ d_j = d_i + \alpha_{ij} \\ 
P_j = i\\ 
\textup{If} \ j \neq t, \textup{put} \ j \ \textup{in Open}\\
\textup{If} \ j = t, \textup{Upper}\ = d_i + \alpha_{ij}\\
3) \ \textup{If Open} \neq 0 \textup{terminate, else to go to} \ (1)
\end{align*}


\end{document}
