\documentclass[11pt, english]{article}
\usepackage[english]{babel}
\usepackage[utf8]{inputenc}



\usepackage{geometry}
 \geometry{
 a4paper,
 left=20mm,
 top=30mm,
 right=20mm
 }


\usepackage{listings}
\usepackage{amsmath}
\usepackage{amsfonts}
\usepackage{amssymb}
\usepackage{amsthm} 
\usepackage{mathrsfs}
\usepackage{mathabx}
\usepackage{graphicx}
\usepackage{eurosym}
\usepackage{subfigure}
\usepackage{dsfont}
\usepackage{bbm}

%\lstdefinestyle{myCustomMatlabStyle}{
%	language=Matlab,
%	numbers=left,
%	stepnumber=1,
%	numbersep=10pt,
%	tabsize=4,
%	showspaces=false,
%	showstringspaces=false
%}

\usepackage{graphicx} 
\usepackage{fancyvrb} 
\usepackage{listings} 

\lstdefinestyle{myCustomMatlabStyle2}{% setup listings 
	language=R,% set programming language 
	basicstyle=\small,% basic font style 
	keywordstyle=\bfseries,% keyword style 
	commentstyle=\ttfamily\itshape,% comment style 
	numbers=left,% display line numbers on the left side 
	numberstyle=\scriptsize,% use small line numbers 
	numbersep=10pt,% space between line numbers and code 
	tabsize=3,% sizes of tabs 
	showstringspaces=false,% do not replace spaces in strings by a certain character 
	captionpos=b,% positioning of the caption below 
	breaklines=true,% automatic line breaking 
	escapeinside={(*}{*)},% escaping to LaTeX 
	fancyvrb=true,% verbatim code is typset by listings 
	extendedchars=false,% prohibit extended chars (chars of codes 128--255) 
	literate={"}{{\texttt{"}}}1{<-}{{$\leftarrow$}}1{<<-}{{$\twoheadleftarrow$}}1 
	{~}{{$\sim$}}1{<=}{{$\le$}}1{>=}{{$\ge$}}1{!=}{{$\neq$}}1{^}{{$^\wedge$}}1,% item to replace, text, length of chars 
	alsoletter={.<-},% becomes a letter 
	alsoother={$},% becomes other 
	otherkeywords={!=, ~, $, *, \&, \%/\%, \%*\%, \%\%, <-, <<-, /},% other keywords 
	deletekeywords={c}% remove keywords 
}

\newcommand{\grafico}[5]{
\begin{figure}
[h!tbp]
\centering
\includegraphics[scale=#2, angle=#3]{#1}
%\captionsetup{width=13cm}
\caption{#4\label{#5}}
\end{figure}
}

\newcommand{\su}[2]{\sum\limits_{#1}^{#2}}


\setlength{\parindent}{0pt}

\title{Stochastic Models and Optimization: Problem Set 1}
\author{Roger Garriga Calleja, José Fernando Moreno Gutiérrez, David Rosenfeld, Katrina Walker}
\date{\today}

\begin{document}
\maketitle

\textbf{Problem 3. Clustering:} \textit{We have a set of $N$ objects, denoted $1, 2, . . . , N$, which we want to group in clusters that consist of consecutive objects. For each cluster $i, i + 1, . . . , j$, there is an associated cost aij. We want to find a grouping of the objects in clusters such that the total cost is minimum. Formulate the problem as a shortest path problem, and write a DP algorithm for its solution.}

The primitives of the problem are:\\
$x_k$ is the last node of a cluster, with $x_k \in S = {0, 1, ..., N}$ for $k = 0, 1, ..., N$\\
$u_k$ is the decision made at every step k over all objects i such that $i \geq x$.\\
$a_{ij}$ is the cost of a cluster running from i to j.\\
\\
Dynamics:\\
$x_{k+1} = u_k$ and\\
$x_0 = 0$\\
\\
$u_k \U_k(x) = {i \in S | i\geq x}$ if $x \neq N$ for $k = 0, 1, ..., N-1$ and\\
$u_k \in U_k(x) = {N}$ if $x= N$\\
\\
$g_k(x, u) = a_{x+1,u}$ if $x \neq N$ for $k = 0, 1, ..., N-1$, and \\
$g_k(x, u) = 0$ if $x = N$\\
\\
We then set up the DP algorithm as follows:\\
$J_N(N) = 0$\\
$J_k(i) = \displaystyle \min_{j \in S|j \geq i}[a_{i+1,j} + J_{k+1}(j)]$ if $x \neq N$ and for $k = 0, 1, ..., N-1$\\
$J_k(i) = 0$ if $i = N$\\
Return $J_0(0)$ as the lowest cost.

\textbf{Problem 5. TSP Computational Assignment:}\\\textit{ Visit the website: http://www.math.uwaterloo.ca/tsp/world/countries.html.
Solve the Traveling Salesman Problem for Uruguay based on the dataset provided. You can use your favorite programming language and solution method for the TSP. Provide a printout of your code with detailed documentation, and compare the optimal solution you obtain to the one available at the website.}\\

The code has been done in R. We used 3 heuristic approaches to find approximate the problem: The nearest neighbor, the greedy algorithm and the simulation anneling. We can see that the best approach (anneling) is above the optimal solution by 12\%, however comparing to the second best it just 1\% below. Furthermore, this 1\% represented an important loose in terms of efficiency. In the following table you can see some important results:

\begin{table}[ht]
	\centering
	\begin{tabular}{rrrrr}
		\hline
		& \textbf{optimal} & \textbf{nearest neighbor} & \textbf{greedy} & \textbf{anneling} \\ 
		\hline
		\textbf{distance} & 79114.00 & 100056.45 & 89559.29 & 88985.51 \\ 
		\textbf{distance/optimal} &  & 1.26 & 1.13 & 1.12 \\ 
		\textbf{run time (min)} &  & 0.19 & 2.55 & 11.69 \\ 
		\hline
	\end{tabular}
\end{table}

\lstset{style=myCustomMatlabStyle2}

\lstinputlisting[language=R]{TSP.R}

\end{document}
